\input{UCLHeader.tex}
\input{UCLCommands.tex}
\begin{document}
\title{Quantum Dots \& Wires - Summary}
\author{Sofia Qvarfort}
\maketitle

\tableofcontents

\section{Elzerman's Lectures}
\subsection{Motivation}
\begin{description}
\item[Core point] It is not about building the best qubit using different approaches, but about where each specific qubit excels. 

\item[Advantages of optically active dots and wires]  include the fact that they
\begin{itemize}
\item Allows for light-matter interface 
\item Use matter qubits to generate specific states of light
\item Use light to control qubits
\end{itemize}

\end{description}
\subsubsection{Semiconductor band structure}
\begin{description}
\item[Diamond lattice structure] given by 
\beq
\vv{R} = n_1 \vv{a}_1  + n_2 \vv{a}_2 + n_3 \vv{a}_3
\eeq

Can also be described by a unit cell with lattice constant $a$. 

\item[Zincblende lattice] group of GaAs, and InAs. Same as diamond lattice. 

Crystal lacks inversion symmetry, which leads to piezoelectric behaviour. 

\item[Crystal translational symmetry] means that
\beq
U(\vv{r} + \vv{R}) = U(\vv{r})
\eeq

\item[Wavefunction in crystal potential] are given by 
\beq
\psi_k ( \vv{r}) = e^{i \vv{k} \cdot \vv{r} } u_k
\eeq

where $u_k(\vv{r})$ a the periodic Bloch part of the wavefunction. 

\item[Bloch electrons] behave almost like free electron, but with modified effective mass. 
\beq
E = \frac{\hbar^2k^2}{2m^*}
\eeq
where
\beq
m^* = \hbar^2 \left(\frac{d^2E}{dk^2} \right) ^{-1}
\eeq


Question: I always forget how to actually solve this equation. 

Attempt at answer:  I think the energy used here is actually different. 

\item[Efect of crystal potential] is to lift degeneracies, like open gaps, between bands 0 and 1, or 1 and 2. Etc. 


\item[Band formation] Bands are formed by bringing atoms close together. The discrete energy levels will interfere until there are efectively an infinite number of energy levels, which form continuous bands.

\item[Covalent solids] (which includes Si, GaAs, InAs) the VB is completely occupied. 	The CB 	is completely empty. 

\item[Conduction band properyties] include
\begin{itemize}
\item has s-like symmetry
\item Derives from anti-bonding s-states since orbital angular momentm $l = 0$ leads to $u_{nk}(\vv{r})$ 
\item Completely empty for covalent solids
\item Weak spin-orbit interaction but still non-zero
\end{itemize}

\item[Valence band properties]
\begin{itemize}
\item p-like symmetry
\item Completely filled for covalent solids
\item Derives from bonding p-states with angular momentum $l = 1$, which leads to $u_{kn}(\vv{r})$ having p-like symmetry
\end{itemize}

Question: What is meant by p-like and s-like symmetry?

Answer: It refers to the parity of the state - it is symmetric or anti-symmetric? I think that s-states are completely symmetric (spheres) whereas p-like states only have one kind of symmetry. And also, as mentioned, it refers to the angular momentum of the state. 

\item[Properties of GaAs and InAs] include
\begin{itemize}
\item Direct bandgap
\item Non-degenerate CB states
\item Degenerate VB states around $\Gamma$ point
\item The degenerate states are heavy holes and light holes around $\Gamma$ point
\item Can neglect split-off band
\item Electrons and holes lower their energy by moving from GaAs into InAs
\item GaAs quantum dot in InAs is optically active
\item InAs lattice constant $a_{InAs} = 1.7 a_{GaAs}$
\end{itemize}

\item[Direct bandgap] occurs if the minimal energy state in the CB and the maximal energy state in the VB have the same $\vv{k}$ vectors. This means that the electron and hole can readily recombine to emit a photon. 

\item[Indirect bandgap] occurs if the minimal energy state in the CB and the maximal energy in the VB have different $\vv{k}$ vectors. Here, the electron and hole cannot recombine to emit a photon, without first transferring some momentum to the crystal lattice. 

Question: Exactly why must the momenta match up? I would have thought that any additional momenta would go into the energy of the photon, like in high energy physics. However, maybe the mechanism here is somewhat different. 

\item[$\Gamma $ point] is the point where $\vv{k} = 0$. It is the middle of the Brillouin zone. This is where we get the holes, for maximum and minimum energy for electrons and holes. 


\item[Split-off band] is a band below the VB. We rarely need to consider it. 

\end{description}
\subsubsection{Spin-Orbit Interaction}
\begin{description}
\item[Adding angular momenta] is done with 
\beq
\vv{J} = \vv{L} + \vv{S}
\eeq
where $\vv{L}$ is orbital angular momentum and  $\vv{S}$ is spin angular momentum. 

\item[Spin-Orbit Hamiltonian] given by 
\beq
H_{so} = \lambda_{so} \vv{L} \cdot \vv{S}
\eeq
where $\lambda_{so}$ is a coupling constant that depends on geometry and material. 

\item[Hamiltonian Eigenstates] We can work around the $\Gamma$ point. There, we have electrons and holes with degenerate energy. We find that the eigenstates of the $\vv{S}$ operator are
\beq
\ket{\frac{1}{2}, \frac{1}{2} } = \ket{\uparrow}
\eeq
\beq
\ket{\frac{1}{2}, - \frac{1}{2}} = \ket{\uparrow}
\eeq

I think that we are here using the first vector in the ket to denote the angular momentum \emph{relative to which} we define the spin direction. The first vector in the ket does not actually determine the eigenstate. This analysis is based on what we see further down. 

Eigenstates of $\vv{L}$  are givne by 
\beq
\mbox{CB (s-states)} =  \ket{00} 
\eeq
\begin{align}
\mbox{VB (p-states)} &= \ket{1,1} && = - \frac{1}{\sqrt{2}} \ket{x + iy} \\
&=\ket{1,0} &&= \ket{z} \\
&= \ket{1,-1} &&= \frac{1}{\sqrt{2}} \ket{x - iy}
\end{align}

Question: I presume here that
\beq
\ket{x + iy} = \ket{+ } + i \ket{i+}
\eeq
or something. 

\item[Eigenstates of total angular momentum] are given by

\beq
CB: = \ket{\frac{1}{2}, \frac{1}{2} }  = \ket{0,0} \ket{\uparrow}
\eeq
\beq
CB: = \ket{\frac{1}{2}, - \frac{1}{2}} = \ket{0,0} \ket{\downarrow}
\eeq
\beq
HH: \ket{\frac{3}{2}, \frac{3}{2} }= \ket{1,1} \ket{\uparrow} 
\eeq
\beq
HH: \ket{\frac{3}{2}, - \frac{3}{2}} = \ket{1,-1} \ket{\downarrow}
\eeq
\beq
LH: \ket{\frac{3}{2}, \frac{1}{2}} = \frac{1}{\sqrt{3}} \left( \ket{1,1} \ket{ \downarrow } + \sqrt{2} \ket{1,0} \ket{\uparrow} \right)
\eeq
\beq
LH: \ket{\frac{3}{2} , - \frac{1}{2}} = \frac{1}{\sqrt{3}} \left( \ket{1,-1} \ket{\uparrow} + \sqrt{2} \ket{1,0} \ket{\downarrow} \right)
\eeq
\beq
SO: \ket{\frac{1}{2}, \frac{1}{2}} = \frac{1}{\sqrt{3}} \left( \ket{1,0} \ket{\uparrow}  - \sqrt{2} \ket{1,1} \ket{\downarrow} \right)
\eeq
\beq
SO: \ket{\frac{1}{2}, - \frac{1}{2}} = \frac{1}{\sqrt{3}} \left( \ket{1,0} \ket{\downarrow} - \sqrt{2} \ket{1, -1} \ket{\uparrow } \right)
\eeq


Question: I'm sure there is logic behind these assignments, but I don't see it. 

\item[Control over electron spin] can be done through the spin-orbit interaction, by controlling the orbital angular momentum $\vv{L}$ through optical transitions. That is, we can use electric fields to change $\vv{L}$ and thus control $\vv{S}$ through the coupling. 

The disadvantage is that the quantum dot becomes more sensitive to photon noise. 

\item[Connection between CB and HH] they both share $m_j$ and $m_s$ projections. This will cause simple optical selection rules. 

\item[Connection between SO and LH] they feature mixtures of spin projections for each $m_j$, which means that optical selection rules will be more complicated. 

\item[Small QDs] see effects of mixed LHs and HHs. 



\end{description}

\subsubsection{Strain}
\begin{description}
\item[Unstrained QDs] have degenerate energies for HHs and LHs. 

\item[Compressive strain] Raises the energy band of HHs. This means that they get lower energy. 

Question: I thought that the higher a band was, the higher the energy... evidently not. 

\item[Tensile strain] raises the LH band (decreasing its energy) and lowers the HH band. 

\item[Source of strain] confinement potential from QD. 

\end{description}

\subsubsection{Quantum Confinement}
\begin{description}

\item[No confinement] Can use this treatment for a bulk material with periodic boundary conditions. 

Grid cell wavevector given by 
\beq
(dk)^3 = \frac{(2\pi)^3}{V}
\eeq
Number of allowed state is
\beq
N(k) = \frac{g_s 4\pi k^2\Delta k}{(dk)^3}
\eeq
with $g_s = 2$ for spin degeneracy, means that the density of states is 
\beq
D_{3D}(k) = \frac{N(k)}{(\Delta k \cdot V)} = \frac{k^2}{\pi^2}
\eeq
Energy is given by 
\beq
E = E_{CB} + \frac{\hbar^2 k^2}{2m^*}
\eeq
because we are looking at electrons in the VB, which carry both CB energy and additional, free moving energy. 

We can thus transform to the DoS in terms of energy, with
\begin{align}
D_{3D}(E) &= D_{3D} (k) \frac{dk}{dE} \\
&= \left( \frac{2}{\pi} \right)^{1/2} \left( \frac{m^*}{\pi \hbar^2} \right)^{3/2} (E-E_{CB})^{1/2}
\end{align}


\end{description}
\subsubsection{Generating confinement}
\begin{description}
\item[Local charge] creates a quantum well potential. Create this by implanting a deep defect. Hard to contain both electrons and holes becuase defect will have charge. 

\item[Gate electrodes] Apply voltages to metal gate electrodes on the top surface of a semiconductor. This does not confine in all three directions, however. We require a 2DEG produced by a band offset, or similar. This method can confined either electrons or holes. 

\item[Band offsets] By introducing InAs in GaAs or similar, we create an offset in both the VB and CB. These bring the energy gap $E_G$ down. We can trap both the electron and the hole here, which makes recombination easier. This is the grown quantum dot. 

\item[Examples of quantum dots] are
\begin{itemize}
\item NV centres
\item Shallow dopants in semiconductors
\item Molecules
\item QDs grown by Molecular Beam Apitaxy
\item Coloidal QDs grown by chemical processes
\item Using gates on heterostructures for single and double QDs
\item QDs in semiconductor nanowires grown by epitaxy
\item Gate defined QDs in semiconductor nanowires
\end{itemize}


\item[Epitaxy] is when we deposit crystalline layers on a seed substrate through a gaseous or liquid precursor. There are many different ways of doing this. 


\end{description}
\subsubsection{Self-assembled quantum dots}
\begin{description}
\item[Dimensions] typical dimensions of a QD in InAs section is about 20nm$\times$5nm. 

\item[Growing procedure] is
\begin{itemize}
\item Start with GaAs layer
\item Grow InAs layer on top, get spheres
\item Apply partial GaAs overgrowth
\item Annealing process
\item GaAs capping
\end{itemize}
By thinning the dots through the annealing process, we gain some control over emission wavelength. Get blueshift due to increased confinement energy. 

\item[Use of strain for growth] Strain between e.g. GaAs and InAs layers cause quantum dots to form. This is caused by misfit dislocation formation. This can lift the degeneracy between LHs and HHs, which is what we want (I think). 


\item[Conditions for growth] of quantum dots include
\begin{itemize}
\item Correct number of layers
\item Usually temperatures around 400-500 C
\item 
\end{itemize}

\end{description}
\subsection{Optical transitions and selection rules}
\subsubsection{Scattering processes}
\begin{description}
\item[Elastic scattering] Occurs when the incoming light is detuned from resonance
\beq
\omega = \omega_0 + \Delta
\eeq
Shining this light on the system causes it to be excited to a time $\Delta t$ allowed by the Heisenberg uncertainty principles. Once the system decays, it emits phase-coherent photons into all directions with original frequency $\omega$. This is the most common scattering process. 

\item[Stimulated Raman scattering] occurs when we have a three level system with $\ket{g}, \ket{v}, \ket{e}$ in increasing energy, where we have an incoming photon at $\omega_1$, which together with $\omega_2$, I think from an external field, causes an excitation. System can either end up in $\ket{v}$ which is a an excitation of the vibrational modes of the lattice, or in $\ket{g}$. These two processes are called Stokes and anti-Stokes. 

Question: I think they are really much more advanced than we are told here. They depend on the available vibrational degrees of freedom of the lattice atoms. 

\end{description}

\subsubsection{Light-matter interaction}
\begin{description}
\item[Semi-classical approach] treats the particles as quantum objects and the light wave as classical. EM wave given by 
\beq
\vv{E}(\vv{r}, t) = \vv{E}_0 \cos{(\omega + \vv{k} \cdot \vv{r})}
\eeq
It drives an electron at $\vv{r}$ around equilibrium position $r_0$. 

\item[Dipole treatment] the electronic oscillations correspond to an oscillating dipole
\beq
\vv{p}(t) = q \vv{r}(t)
\eeq


\item[Classical energy of dipole in E-field] is given by 
\beq
U = - \vv{p} \cdot \vv{E} = - q \vv{r}(t) \cdot \vv{E}(\vv{r}, t)
\eeq

\item[Long wavelength approximation] The wavelength of the photon is much larger than any distance the electron is expected to move over the timespans that we are considering. Thus $E(\vv{r}, t)= E(r_0, t)$. 

\item[Semi-classical Hamiltonian] 
\beq
H _{int} = - q \hat{p} \cdot \vv{A}( \vv{r}_0, t)/m
\eeq

Question: This comes from the the addition of $\vv{A}$ to the kinetic energy term, which in turn comes from the Lorentz force. But where did our electric field go? 

\end{description}
\subsubsection{Properties of self-assembled quantum dots}
\begin{description}
\item[Artifical atom] simulated by having only a few electrons occupy one space. 
This can be done by using E-fields, gates or material interfaces. 

\item[Desired properties] are
\begin{itemize}
\item Low coupling to noise
\item Small capacitance to give large
\beq
E_c = \frac{e^2}{C} > k_B T
\eeq
\item Ability to cool system to energies below that of energy spacing
\beq
\Delta \epsilon > k_B T
\eeq
\end{itemize}

\item[Optically active quantum dot] can be made by confining the electrons and holes in the same spaces. We also have to make sure electron-hole combination is not forbidden by selection rules. 

\item[Controlling an OAQD] We first load the QD with one electron, then we make sure the material couples the electron spin to a laser via the spin-orbit interaction. 

\end{description}
\subsubsection{Exciton selection rules}
In this section we consider the $X^0$ exciton, which is made up of one electron and one hole. 

\begin{description}
\item[Bright exciton states] for the $X^0$ exciton are given by 
\beq
\ket{\uparrow \Downarrow} , \ket{\downarrow, \Uparrow}
\eeq
These can decay to the ground state via an emission of a photon. This is because of the selection rules. There is a $\Delta m_z = \pm 1$ selection rule, which means that the system must change in one unit of angular momentum in the $z$-direction. 

\item[Exciton state couplings] We have that 

\beq
\sigma_- \ket{\uparrow \Downarrow} \rightarrow \ket{0}
\eeq
\beq
\sigma_+ \ket{\downarrow \Uparrow} \rightarrow \ket{0}
\eeq

Note that if we couple to these states in order to create the $XX^0$ biexciton state, we \emph{create} the state that the photon couples to. This is because we are not coupling the existing state to the ground state, but we get one of the existing states to obtain another set of electron and hole. 

\item[Energy of charge exciton transitions] Are given by considering the energies and potentials created by electrons and holes. 

Single exciton: \\
\beq
E(1e, 1h) = E^e + E^h - V^{eh}
\eeq

Biexciton $X^{1-}$: \\
\beq
E(2e, h) 2E^e + E^h + V^{ee} - 2V^{eh}
\eeq

Biexciton $X^{2-}$: \\
\beq
E(3e, 1h) = 3E^e + E^h + 3V^{ee} - 3V^{eh} + \epsilon_{sp}
\eeq
where $\epsilon_{sp}$ is the difference between singlet and triplet configurations. 

Biexciton $XX^0$: \\
\beq
E(2e, 2h) = 2E^e + 2E^h + V^{ee} + V^{hh} -4 V^{eh}
\eeq

\item[Emission photon energy] can be found by comparing the energy of the exciton with the energy of the biexciton. For example, the energy emitted by a $\sigma_+	$ photon during $XX^0 \rightarrow X^0$ is given by 
\beq
\Delta E(XX^0\rightarrow X^0) = E(2e, 2h) - E(1e, 1h) = E^e + E^h + V¨{ee} + V^{hh} - 3V^{eh}s
\eeq



\item[Interaction energy difference] we find that
\beq
V^{ee} < V^{eh} < V^{hh}
\eeq
This follows because electrons and heavy holes have different effective mass, and we know that $m^*_e < m^*_h$, hence the name heavy hole. A larger effective mass means stronger confinement. This in turn means that the wavefunction is more spread-out, causing the electrons to be further apart on average. This means that $V^{ee}$ is smaller than $V^{hh}$, even though they have the same charge. 

\item[Resonance Fluorescence] is when the laser is tuned to resonate with the QD. Drives QD in elastic light scattering. The incoming wavelength plus the QD emission wavelength are scattered. We then suppress detected light with a polariser and use single-photon detection on the rest. 

\item[Resonant reflections] tune the laser to resonance, but amplify resulting light. Measure interference between laser light and QD emission with homodyne detection. 


\end{description}

\subsection{QDs as qubits}
\begin{description}
\item[General idea] we focus on the $X^{1-}$ state. One electron is the qubit, and the remaining electron-hole pair can be used for qubit manipulation. 

\item[Control] can be done by embedding the qubit between a back contact and a gate. By tuning the voltage of the gate, we control the vertical electric field. 

\item[Qubit transitions] the transition $\ket{\uparrow} \rightarrow \ket{\downarrow}$ of the qubit is forbidden due to selection rules. But using the trion state $X^{1-}$ they are available. 

\item[Advantages] very fast manipulation through lasers due to large optical energies. Polarisation selection rules means we can access particular spin states, even when they are degenerate. 

\item[Presence of B-field] lifts the degeneracy of the electron states. The direction of the magnetic field determines the ground states. Basically, the B-field generates a Hamiltonian. 

\item[Faraday configuration] means the B-field is along the growth direction. Get eigenstates $\ket{\uparrow}$ and $\ket{\downarrow}$ along $z$. Similarly, get trion eigenstates along $z$: $\ket{T_\Uparrow}$ and $\ket{T_\Downarrow}$. 

Note that this also breaks the degeneracy between  $\ket{\uparrow \Downarrow \downarrow}$ and $\ket{\uparrow \Uparrow \downarrow}$. This is because of the Zeeman interaction. 

\item[Zeeman energy splitting] The two trion states will be split by 
\beq
\Delta E^h_Z = g^h \mu_B B
\eeq
and the two qubit states will be split by 
\beq
\Delta E^e_Z = g^e \mu_B B
\eeq

Question: Why aren't the trion states split by the electron Zeeman interaction?

Answer: Presumably because the $\ket{\uparrow \downarrow}$ combination is degenerate. 

\item[Voigt configuration] here we apply the B-field perpendicular to the growth direction. The ground states are
\beq
\ket{+} = \frac{1}{\sqrt{2}} \left( \ket{\uparrow } + \ket{\downarrow} \right)
\eeq
\beq
\ket{-} = \frac{1}{\sqrt{2}} \left( \ket{\uparrow} - \ket{\downarrow} \right)
\eeq
The same goes for the trion states. Basically, we can decide which Hamiltonian and thus energy eigenstates we get by changing the magnetic field direction. 

\newpage
\end{description}
\section{Transport}

\subsection{Conductance of Ballistic conductor between reflectorless contacts}
\begin{description}
\item[Conductance] Describes how easy it is for a current to travel through a material. Classically given by 
\beq
R = \sigma \frac{vA}{L}
\eeq

\item[Derivation of current] Start with the current $I = env$ where $n$ is the number of particles and $v$ is the velocity of the particles. This holds for uniform electron gas of $n$ electrons. 

Consider then $+k$ modes. We denote the current $I^+$ and the distribution they follow $f^+$. We could do the same for $-k$ modes, but we shall do so later. Let the electrons be distributed according to 
\beq
n = \frac{f^+(E)}{L}
\eeq
where $L$ is the length of the conductor. Then, the current over all $k$ modes will be
\beq
I^+ = \frac{e}{L} \sum_k vf^+(E) = \frac{e}{L} \sum_k \frac{1}{\hbar} \frac{\p E}{\p k } f^+ (E)
\eeq
where
\beq
v  = \frac{1}{\hbar} \frac{\p E}{\p k}
\eeq
Then, we must convert the sum into an integral. This is done as follows:
\beq
\sum_k \rightarrow 2 \times \frac{L}{2\pi} \int \mathrm{d}k
\eeq

Question: It is not clear to me how this conversion happens. I presume that you need $L$ to cancel out the units of $\mathrm{d} k $. 

Thus we obtain 
\beq
I ^+ = \frac{2 e}{h} \int^\infty_\epsilon f^+(E) \mathrm{d} E
\eeq
where $\epsilon$

\end{description}

\newpage
\section{Mark Buitelaar's Lectures - Quantum Transport}
\subsection{Introduction}
\begin{description}
\item[Quantum Dot] a 0-dimensional structure which contains one hole and/or electron. It can e.g. be used for quantum information processing. By making use of quantum properties, an electron can be captured and manipulated in a quantum dot. 

\item[Quantum Wire] a 1-dimensional structure, often a carbon nanotube, where quantum effects influence the electron transport properties. 

\item[Some examples] of quantum dots and wires include
\begin{itemize}
\item Graphene 
\item Carbon nanotubes
\item Si/SiGe heterostructures 
\item Si nanowires
\item GaAs based quantum dots
\item InAs quantum dots
\end{itemize}

\item[Probing properties] is often done by simply connecting the quantum wire or dot to an external voltage and then measuring the other electrical properties. 

\item[States in a waveguide] which we have studied well can be used to model the behaviour of quantum states inside a wire, have energy
\beq
E_n(k_x) = \frac{\hbar^2 k_x^2}{2m} + \frac{\pi^2 \hbar^2}{2m} \left( \frac{n_y^2}{a^2 } + \frac{n^2_z}{b^2} \right)
\eeq
where $a$ is the width of the waveguide in the $z$ direction and $b$ in the $y$ direction. 


\end{description}
\subsection{Conductance from transmission}
\begin{description}
\item[Ballistic conductor] has no resistivity from the scattering of electrons. A ballistic conductor differs from a superconductor by the absence of the Meissner effect. The only scattering process that occurs is when electrons scatter off the walls or the ceiling of the conductor, otherwise they follow Newtons 2nd law of motion. 

\emph{Properties}: \\
- $T(E) = 1$ \\
- 

\item[Boltzmann transport equation] can be used to write down an expression for the current. We find
\beq
I = \frac{2e}{h} \int^{\infty}_{- \infty} f^+(E)M(E) T(E)\intd E
\eeq
where $f^+(E)$ is the deviation form the equilibrium distribution (it is the perturbation), $M(E) $ is the number of propagating modes in the channel and $T(E)$ is the transmission probability, which for a ballistic conductor has $T = 1$. 

\item[Ballistic conductor current] if we insert the Fermi energy $\mu_1$ and $\mu_2$ in the integral limits, we find that
\beq
I = \frac{2e^2}{h} M \frac{\mu_1 - \mu_2}{e} 
\eeq
where we (I think) have assumed that $M(E)$ is independent of the energy. 

\item[Current formula derivation] Start by considering a single mode $M = 1$. The occupation of the states with $k$ is determined by the Fermi function $f^+(E)$ (which I presume is the Fermi-Dirac distribution function). Approximate the electrons in the conductor to be a free-moving gas. If the gas is uniform and has a density of $n$ electrons per unit length, they will carry a current if all electrons move with velocity $v$ (modulated by the Fermi distribution). The current is then given by 
\beq
I = env
\eeq
Let the electron density of a single $+k$ state be $n = 1/L$, where $L$ is the length of the conductor. Then, the current is given by 
\beq
I = \frac{e}{L} \sum_k vf^+(E)
\eeq
We can then obtain the velocity from the dispersion relation, such that 
\beq
v = \frac{1}{\hbar}\frac{\p E}{\p k}
\eeq
and by turning the sum into an integral, through the following way
\beq
\sum_k \rightarrow 2 \frac{L}{2\pi} \int \intd k
\eeq

Question: How can I justify this transformation? 
\beq
I = \frac{2e}{h} \int^{+\infty}_\epsilon f^+(E) \intd E
\eeq


\item[Contact Conductance] from the above expression for the current, we identity
\beq
 G_C = \frac{2e^2}{h} M
\eeq
which is the conductance. This is modified for a non-ballistic conductor by multiplying it by the transmission probability $T$. M

Note that the contact conductance increases in steps as the number of modes $M$. Note also that the resistance does not scale with length or area of the conductor. 

Question: Then how do we determine the number of modes? Surely this is a property of the material itself? 


\item[Contact Resistance] the resistance is the inverse of the conductance. We find that
\beq
R = G_C^{-1} = \frac{h}{2e^2 M} \sim 12.9 k \Omega /M
\eeq

\item[Difference in Fermi energy] as can be seen for the formula for the current, we find that the current flow depends on the difference in Fermi energy. All electrons will move around randomly, but only when we have this energy difference will the electrons strive towards a lower equilibrium energy, and therefore move, which in turn carries a current. 

\item[Transverse modes] also called subbands are available to electrons in the conductor. Because of the Pauli exclusion principle, only one electron (with the same spin) can occupy the band at one time. 

Each subband has its own dispersion relation with a cutoff energy, $\epsilon_N = E(N, k=0)$ below which it cannot propagate. Basically, one the mode reaches this energy, the energy just keeps increasing without any changes to $k$. 

\item[Number of transverse modes] can be obtained by counting the number of modes below the cutoff energy. 
\beq
M (E) = \sum_N \theta(E- \epsilon_N)
\eeq
where $\theta$ is the Heaviside step-function, and $N$ is the number of electrons. 

\item[Fermi energy] is given by
\beq
E_F = \frac{\hbar^2 k^2_F}{2m}
\eeq

\end{description}

\subsection{Conductance from Transmission}
\begin{description}
\item[Landauer formula] is given by 
\beq
G = \frac{2e^2}{h} MT
\eeq
This is the general formula for conductance, where we have taken into account the number of modes $M$. 

\item[Resistance] in general resistance is given by 
\beq
G^{-1} = \frac{h}{2e^2M} + \frac{h}{2e^2 M} \frac{1 - T}{T}
\eeq
where we write the formula in this way because the transmission probability adds. 

\item[Conductance with reflection]
Consider having a conductor between two leads. We can calculate the net current $I_2$ by calculating the current difference:
\beq
I_2 = I_1 ^+ - I_1^-
\eeq
Now, consider that $I^+_1$ is the incoming current, and $I^-_1$ is the outgoing current, which is reflected back from the unsuccessful transmissions. We find
\beq
I_1^+ = \frac{2e}{h} MT(\mu_1 - \mu_2)
\eeq
\beq
I_1^- = \frac{2e}{h} M(1-T)(\mu_1 - \mu_2)
\eeq
Thus the difference is
\beq
I = I^+_1 - I_1^- = \frac{2e}{h} MT(\mu_1 - \mu_2)
\eeq

\item[Electrochemical potential] For a conductor connected to two contacts, we can divide it up into areas. If there exists a difference in electrochemical potential $(\mu_1 - \mu_2)$, it will drop in the middle of the conductor. 

\item[Coherent conductor] I have found no definition for this, but assume that a conductor is coherent if we see quantum behaviour. Basically, the phenomena happen on such large (or long) scales that we can observe them without them decohering. 

However, it could equally well be that coherent refers to the way some other phenomena works. 

\item[Coherent conductor interference] Consider two barriers with a conductor in-between them. As an electron travels across the distance, it picks up a phase $e^{i \chi}$. An incoming electron will reflect a number of times between the barriers before being transmitted. 

Insert derivation of probabilities from problem sheet. 

\end{description}

\subsection{Nanotube quantum dot}
In this section, we describe and island lying between a source and a gate. Electrons from the source will hop onto this island, which in turn acts like a capacitor. 

The equivalent circuit for this situation is described by two capacitors $C$ and $C_g$ for the gate in series. 

\begin{description}
\item[Conductance behaviour] with temperature goes approximately as
\beq
G_{peak} \sim \frac{1}{T}
\eeq

\item[Charging energy] is given by 
\beq
E_C = \frac{e^2}{C}
\eeq
where $C$ is the total capacitance of the quantum dot. 

\item[Charge on the island] is quantised. We find
\beq
- q_2 + q_1 = eN
\eeq
where $q_2$ and $q_1$ are the respective charge on the island. 

\item[Total electrostatic energy] is the energy accumulated in the capacitors and the work done by the voltage source to transfer the charge $q_2$ to the gate electrode. 

Question: What role does $q_1$ play? Is it the charge that is already on the island? 

The energy is
\beq
E_{el} = \frac{1}{2} \left( \frac{q^2_1}{C} + \frac{q^2_2}{C_g} \right) - q_2 V
\eeq

\item[Capacitor charge-voltage relationship] is
\beq
CV = q
\eeq
Thus, for the island and the gate, 
\beq
C V_1 = q_1
\eeq
\beq
C_g V_2 = q_2
\eeq


\item[Addition energy] is the charge in the electrochemical potential as electrons are added to a quantum dot. It is given by 
\beq
\Delta E_{add} = \mu(N+1) - \mu(N) = \frac{e^2}{C_\Sigma} + E_{M+1}^K - E_M^K
\eeq
where 

For spin degeneracy, $E_{M+1}^K - E_M^K$ can be zero. 

\item[Electrochemical potential of quantum dot] The electrochemical potential $\mu$ is the difference between the two potentials. Think of the lift between the two energy levels. $\mu$ is different depending which energy level is already occupied.  We have
\beq
\mu(N+1) = E(N+1) - E(N)
\eeq
That is, we compare the energy for the two different levels. 




\end{description}

\subsection{Notes from Problem Sheet}
\begin{description}
\item[Landauer formula] is given by 
\beq
G = \frac{2e^2}{h} MT
\eeq
where $G$ is the conductance. 

\item[Current] is the conductance times the difference in chemical potential
\beq
I = G \frac{\Delta \mu}{e}
\eeq
The current flows to compensate for the chemical potential difference. 

\item[Difference in electrostatic energy] has to be taken between the middle of potential energies for the contacts. 

For one contact with $\Delta \mu = \mu_L - \mu'$ and one with $\mu'' - \mu_R$, we get
\beq
e V = \frac{\mu_L + \mu'}{2} - \frac{\mu'' + \mu_R}{2}  = \frac{\mu_L - \mu_R}{2} + \frac{\mu' - \mu''}{2}
\eeq

\item[Current conservation] bas to occur at the connection of the contacts.  We write
\beq
N(\mu_L - \mu') = N(\mu''- \mu_R) = M(\mu_L - \mu_R)
\eeq

\item[Spin degeneracy in Coulomb diamonds] the large-small-large difference in diamonds reflects spin degeneracy. This is because of the Pauli blockade - we cannot put in another electron in the same state. 

\item[Addition energy from Coulomb diamonds] given by 
\beq
\Delta E_{add} = \mu(N+1) - \mu(N) = \frac{e^2}{C_\Sigma}  + E_{N+1}^K - E^K_N
\eeq
here
\beq
E_{N+1}^K - E^K_N = \Delta E
\eeq
which is the difference between odd and even number of electrons. 

\item[Height of small diamonds] is given by 
\beq
\frac{e^2}{C_\Sigma} = e  V_{SD}
\eeq

From this we can infer the charging energy
\beq
E_c = \frac{eV_{SD}}{2}
\eeq



\item[Height of large diamonds] If the separation between energy levels $\Delta E = E_{N+1}^K - E_N^K$ is known, such as for nanotubes, with 
\beq
\Delta E = \frac{v_F h}{4L}
\eeq
then we can work out the height of the large Coulomb diamonds. 
\beq
V_{SD} = \frac{e}{C_\Sigma} + \frac{\Delta E}{e}
\eeq
for the energy in eV. 


\item[Width of diamonds] is derived from the total capacitance. The width of the diamond is in terms of the gate voltage $\Delta V_g$, we get that
\beq
e \Delta V_g \frac{C_g}{C_\Sigma} = \frac{e^2}{C_\Sigma} + \Delta E
\eeq
Through this, we measure the response of the quantum dot to the applied voltage from the gate. 

\item[Role of gate voltage] we use the gate voltage to control the properties of the quantum dot and decide how many charges there are on the quantum dot at this time. 

If we just changed the $V_{SD}$ we only see spikes when the energy levels align. I guess by changing both $V_{SD}$ and $V_g$, and by doing so  very precisely, we could actually keep a current flowing indefinitely. Or, at least, there would be conductance. 

\item[Size of Coulomb diamonds] You have to apply a larger energy when there are an even number of electrons present on the quantum dot. 


\item[Temperature dependence] When do we no longer see conductance? When the temperature is too high, and thermal excitations can overcome the charging energy. 

For there to be observable conductance, the charging energy needs to be much higher than the thermal energy given by $k_B T$. 

Question: Wouldn't this increase the conductance? 

Answer: I think what i t refers to is the fact that electrons would be able to jump from both directions, and also populate the qubit with more than one electron. Basically, large thermal fluctuations will mess things up. 

\end{description}
\end{document}