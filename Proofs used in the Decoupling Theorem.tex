\documentclass[12pt,a4paper]{article}

\setlength{\parindent}{0 in}
\setlength{\parskip}{0.1 in}
\setlength{\oddsidemargin}{0.25 in}
\setlength{\evensidemargin}{-0.25 in}
\setlength{\topmargin}{-0.5 in}
\setlength{\textwidth}{7.0 in}
\setlength{\textheight}{9.5 in}
\setlength{\headsep}{0.45 in}


\usepackage{amsmath,amsfonts,graphicx}
\usepackage{setspace}
\usepackage{hyperref}
\usepackage[nottoc,numbib]{tocbibind}
\usepackage{tocloft}
\usepackage[outermargin=2 in]{geometry}
\usepackage{scrextend}
\usepackage{tensor}
\usepackage{cancel}
\usepackage{slashed}


%Adding `Appendix' to the appendices
\usepackage[toc,page]{appendix}

%Add a bullet point to description items
\usepackage{enumitem}

\large
\bibliographystyle{unsrt}
%New commands
\newcommand{\beq}{\begin{equation}}
\newcommand{\eeq}{\end{equation}}
\newcommand{\p}{\partial}
\newcommand{\trace}[1]{\mathrm{Tr}\left[#1 \right]}
\newcommand{\bpmat}{\begin{pmatrix}}
\newcommand{\epmat}{\end{pmatrix}}
\newcommand{\vv}[1]{\vec{#1}}
\newcommand{\mat}[1]{\uuline{#1}}
\newcommand{\norm}[1]{\| #1 \|}
\newcommand{\op}[1]{\mathbb{#1}}

\newcommand{\Hilbert}{\mathcal{H}}

\newcommand{\cl}[1]{\mathcal{#1}}

\newcommand{\identity}{\mathbb{I}}

%Pauli
\newcommand{\Xhat}{\hat{X}}
\newcommand{\Yhat}{\hat{Y}}
\newcommand{\Zhat}{\hat{Z}}
\newcommand{\PauliX}{\bpmat 0 & 1 \\ 1 & 0 \epmat}
\newcommand{\PauliZ} {\bpmat 1 & 0 \\ 0 & -1\epmat}
\newcommand{\phihat}{\hat{\phi}}
\newcommand{\xhat}{\hat{x}}
\newcommand{\phat}{\hat{p}}
\newcommand{\Lag}{\mathcal{L}}

%Gell-Mann matrices

\newcommand{\GMone} {\bpmat 0 & 1 & 0 \\ 1 & 0 & 0 \\ 0 & 0 & 0 \epmat }
\newcommand{\GMsix}{\bpmat 0 & 0 & 0 \\ 0 & 0 & 1\\ 0 & 1 & 0\epmat}

%Density matrices
\newcommand{\rhotwo}{\bpmat 1 & e^{-it} \\ e^{it} & 1 \epmat}
\newcommand{\rhothree} {\bpmat 1 & e^{it} & e^{2it} \\
e^{-it} & 1 & e^{it} \\
e^{-2it} & e^{-it} & 1 \epmat}


%Packages
\usepackage{braket}
\usepackage{ulem}
\usepackage{xcolor}
\usepackage[font={small,it}]{caption}

%Puncuation 
\newcommand{\punkt}{\mbox{.}}
\newcommand{\comma}{\mbox{,}}

\def\dbar{{\mathchar'26\mkern-12mu d}}

\newcommand{\Hhat}{\hat{H}}
\newcommand{\ahat}{\hat{a}}
\newcommand{\bhat}{\hat{b}}
\newcommand{\chat}{\hat{c}}
\newcommand{\Phihat}{\hat{\Phi}}

\newcommand{\eq}[1]{$#1$}

%Undertilded quantities
\newcommand{\tildeq}{\underset{^\sim}q}
\newcommand{\tildep}{\underset{^\sim}p}

%Curly letters
\newcommand{\calE}{\mathcal{E}}

\newcommand{\Nhat}{\hat{N}}














\begin{document}
\title{Proofs Needed to Derive the Decoupling Inequality}
\author{Sofia Qvarfort}
\maketitle
\tableofcontents


\section{Proof of the Swap Trick}
Consider four systems, $A, A'$ and $B, B'$. Alice holds two systems whereas Bob holds his two. The Swap operator swaps the states of $B$ and $B'$. That is, we change the labelling of the states. 

Question: What is the physical interpretation of the Swap Trick? Are we physically changing the states? If $A$ is entangled with $B$, is $A$ then entangled with $B'$ once we have swapped the systems? 

The Swap operator $S_{BB'}$ is defined as
\beq
S_{BB'} \ket{\psi}_B \otimes \ket{\phi}_{B'} = \ket{\phi}_B \otimes \ket{\psi}_{B'}
\eeq
Explicitly, we have that
\beq
S_{BB'} = \sum_{ij} \ket{i}\bra{j}_B \otimes \ket{j}\bra{i}_{B'}
\eeq

The lemma we wish to prove is the following. 

\textbf{Lemma: (Swap Trick)} For a quantum state $\rho_B$, we find that
\beq
\trace{\rho_B ^2 } = \trace{(\identity_{AA'} \\otimes S_{BB'} ) (\rho_{AB} \otimes \rho_{A'B'} ) }
\eeq
where $S_{BB'}$ denotes the swap operations. 

Proof: Throughout this proof, we will make use of Einstein index notation. Every repeated index is implicitly summed over. We begin by writing down a general, bipartite quantum state, 
\beq
\ket{\psi}_{AB} = c_{ij} \ket{i}_A \ket{j}_B
\eeq
Similarly, let
\beq
\ket{\psi}_{A'B'} =  d_{ab} \ket{a}_{A'} \ket{b}_{B'}
\eeq
Thus, we can write down two general density matrices:
\begin{align}
\rho_{AB} &= c^\dagger_{ij} c_{kl} \ket{kl}\bra{ij} \\
\rho_{A'B'} &= d^\dagger_{ab} d_{cd} \ket{cd}\bra{ab}
\end{align}
Finally, we also require a state on $B$, 
\beq
\rho_B = \sum_{ij} a_i^*a_j \ket{i}\bra{j}
\eeq


Then, we start by writing down the LHS of the statement we wish to prove. We find that 
\begin{align}
\trace{\rho_{B}'} &= \trace{ \sum_{ikl} a^*_i a_k a^*_k a_l \ket{i}\bra{l} } \nonumber \\
&= \sum_{ijkl} \bra{j} a_i^* a_k a_k^* p_j \ket{i}\braket{l|j} \nonumber \\
&= \sum_{ijkl} a_i^* a_k a_k^* a_j \delta_{ij} \delta_{lj} \nonumber \\
&= \sum_{ik} |a_i|^2 |a_k|^2
\end{align}

Then, let us apply the Swap operator to the other states. 
\begin{align}
\trace{( \identity_{AA'} \otimes S_{BB'} )(\rho_{AB} \otimes \rho_{A'B'} )} \nonumber \\
&= \trace{(\identity_{AA'} \otimes S_{BB} ) \left( \sum_{ijkl,abcd} c_{ij}^\dagger c_{kl} d^\dagger_{ab} d_{cd} \ket{klcd}\bra{ijab} \right)  \nonumber } \\
&= \trace{\sum_{ijkl,abcd} c_{ij}^\dagger c_{kl} d^\dagger_{ab} d_{cd} \ket{kjcd}\bra{ilab} } \nonumber  \\
&= \sum_{ijkl,abcd,xywz} \bra{xywz} \left( c_{ij}^\dagger c_{kl} d^\dagger_{ab} d_{cd} \ket{kjcd}\bra{ilab}  \right) \ket{xywz} \nonumber \\
&= \sum_{ijkl,abcd,xywz} \delta_{xk} \delta_{yj} \delta_{wc} \delta_{zd} c_{ij}^\dagger c_{kl} d_{ab} d_{cd} \delta_{ix} \delta_{ly} \delta_{aw} \delta_{bz} \nonumber \\
&= \sum_{ijab} |c_{ij}|^2 |d_{ab}|^2 
\end{align}
We therefore see that the two expressions are the same. This holds because we can map each index in the two-index sum to two of the indices in the sum above. 

 
\section{Proof of the Transpose Trick}
\section{Proof of Uhlman's Theorem}

\end{document}